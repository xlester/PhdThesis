\chapter{Fringe pattern demodulation}

\section{Introduction}

A fringe pattern is defined as a sinusoidal signal where a continuous map, 
analogous of the physical quantity being measured, is phase-modulated by an 
interferometer, Moire' system, etc. An ideal stationary fringe pattern is usually
modeled by
\begin{equation}\label{eq:FringePattern}
I(x,y)=a(x,y)+b(x,y)cos[\phi(x,y)],
\end{equation}
where $a(x,y)$ and $b(x,y)$ are the background and local contrast functions,
respectively; and $\phi(x,y)$ is the searched phase function.

Analyzing Eq. \eqref{eq:FringePattern} one can see that the phase function
$\phi(x,y)$ cannot be directly estimated since it is screened by two other
functions, $a(x,y)$ and $b(x,y)$. Additionally, $\phi(x,y)$ can only be
determined modulo $2\pi$ because the sinusoidal fringe pattern $I(x,y)$ depends
periodically on the phase ($2\pi$ phase ambiguity); and its sign cannot be 
extracted from a single measurement without a priori knowledge (sign ambiguity)
due to even character of the cosine fuction [$cos(\phi)=cos(-\phi)$]. Finally, 
in all practical cases some noise $\eta(x,y)$ is introduced in an additive and/or
multiplicative way, and the fringe pattern may suffer from a number of distortions,
degrading its quality and further screening the phase information [142,18].

It must be noted that even if careful experimental setups could prevent the
screening of $\phi(x,y)$ due to the unknown signals $a(x,y)$, $b(x,y)$ and $\eta
(x,y)$, one would still have to deal with the sign ambiguity and the $2\pi$ phase
ambiguity. Because of these ambiguities, the solution for this inverse problem is
not unique; this is because several phases can produce exactly the same sinusoidal
signal.

If we rewrite Eq. \eqref{eq:FringePattern} by means of the complex representation of
the cosine function
\begin{equation}\label{eq:FringePatternExp}
 I(x,y)=a(x,y)+ \frac{1}{2} b(x,y)[e^{i \phi(x,y)} + e^{-i \phi(x,y)}],
\end{equation}
and one is able to isolate one of the analytic signals in Eq. 
\eqref{eq:FringePatternExp}, lets say $\frac{1}{2} b(x,y)e^{i \phi(x,y)}$, we have
\begin{equation}\label{eq:TanPhi}
 \tan \hat{\phi}(x,y)=\frac{Im\{(1/2)b(x,y)e^{i \phi(x,y)}\}}{Re\{(1/2)b(x,y)
e^{i \phi(x,y)}\}}.
\end{equation}
Computing the arc-tangent of the above formula one obtains a wrapped estimation of
the phase under study: $\phi(x,y)$ mod $2\pi$. Thus, the final step of this fringe
pattern demodulation process usually involves an additional phase unwrapping 
process. Nevertheless, when working with good quality data this last step is 
straightforward. Next we will illustrate the easiest way to obtain these analytic 
signals.

\section{Fringe pattern temporal carrier}

A fringe pattern obtained as the output of a measuring system may be modified
by the optoelectronic-mechanical hardware (sensors and actuators) and software
(virtual sensors and actuators) of the system [83]. With these modifications one
is able to introduce known changes in the argument of the sinusoidal signal:
\begin{equation}\label{eq:FringePatternTemporal}
 I(x,y,t)=a(x,y)+b(x,y)cos[\phi(x,y) + \c{C}(x,y,t)],
\end{equation}
where $\c{C}(x,y,t)$ is a known function (typically a reference plane) and it is 
called the spatio-temporal carrier of the interferogram. The spatial and/or 
temporal carriers are of extreme importance in modern interferometry: first of 
all, its presence allows to solve the sign ambiguity since in general 
$\cos(\phi+\c{C})\not=\cos(-\phi+\c{C})$. They also allow to isolate the analytic signal
$\frac{1}{2} b(x,y)e^{i \phi(x,y)}$. In phase-shifting interferometry (PSI) the
linear temporal carrier can be written as:
\begin{equation}\label{PSIcarrier}
 \c{C}(x,y,t)=\alpha k,
\end{equation}
where $\alpha$ is the temporal lineal carrier, typically $\alpha=\pi/2$, and $k$
is the discret temporal variable. The independent temporal variable $k$ represents
the $k$-frame of the PSI sequence. In this context, knowing the temporal carrier 
$\alpha$ (phase step as known in PSI), the objective of the PSI demodulation 
methods is to estimate the complex field
\begin{equation}\label{eq:Complexfield}
 f(x,y)=\frac{1}{2}b(x,y)e^{i\phi(x,y)},
\end{equation}
of the interferometric signal at the $(x,y)$ pixel. Then, the phase at $(x,y)$ 
is obtained as:
\begin{equation}\label{eq:PSIphase}
  \hat{\phi}(x,y)=arg[f(x,y)]=\arctan\left(\frac{\sin[\phi(x,y)]}{\cos[\phi(x.y)]}
\right),
\end{equation}
which is an alternative form of Eq. \eqref{eq:TanPhi}.

In the following sections we will analyze several methods to estimate the 
analytic signal $f(x,y)$.

\section{Phase-shifting algorithms (PSAs)}

In general, a phase-shifting algorithm (PSA) can be described as quadrature linear
filter which is completely characterized by its impulse response function, $h(k)$,
or by its frequency transfer function (FTF) in the Fourier domain, $H(\omega)$:
\begin{equation}
 h(t)=\sum^{N-1}_{n=0} c_n \delta(k-n),
\end{equation}
\begin{equation}
 H(t)=\sum^{N-1}_{n=0} c_n e^{-i\omega n},
\end{equation}
where ${c_n} \in \mathbb{C}$. In order to be a valid PSA, this FTF must fulfill
the so-called quadrature conditions:
\begin{equation}
 H(0)=H(-\alpha)=0, \quad H(\alpha)\not=0.
\end{equation}

Thus, the application of this quadrature filter produces the following analytic
signal for $k = N-1$ (where all the available data is involved):
\begin{equation}
 A_0(x,y)e^{i\phi}=\sum^{N-1}_{n=0} c_n I(x,y,n).
\end{equation}

From the above analytic signal, we obtain the searched phase $\phi(x,y)$, modulo
$2\pi$, by computing its angle as Eq. \eqref{eq:PSIphase}. For completeness, 
solving for $\phi(x,y)$ is found the so-called arc-tangent formulation of a PSA:
\begin{equation}
 \hat{\phi}(x,y) mod 2\pi= \arctan \Bigg(\frac{Im\{c_0I_0+c_1I_1 + \dots + c_{N-1}
I_{N-1}\}}{Re\{c_0I_0+c_1I_1 + \dots + c_{N-1} I_{N-1}\}}  \Bigg)
\end{equation}
The above results are obtained under the assumption that the quadrature filter
and the temporal samples of the interferogram are perfectly tuned at the same
frequency $\alpha$ (radias per sample). When the actual sampling frequency is
$\alpha+\Delta$, that is when some detuning error occur, the resulting phase 
estimation is given by
\begin{equation}
 \hat{\phi}(x,y)=\phi(x,y)-D(\Delta)\sin[2\phi(x,y)],
\end{equation}
where the amplitud of the detuning-error is given by the ratio
\begin{equation}
 D(\Delta)=\frac{|H(-\alpha-\Delta)|}{|H(+\alpha+\Delta|},
\end{equation}
for $|\Delta/\alpha|<<1$.This equation shows that detuning-error causes the 
estimated phase to be distorted by a component with twice the original fringes 
frequency. Moreover, actual signals are always distorted by some amount of random
noise so a more cautious approach it is to consider additive white-noise 
distortion (with random components in every spectral frequency). Under this 
assumption, when filtering with quadrature linear filters, the signal-to-noise 
power ratio gain of the PSA is given by:
\begin{equation}
 G_{S/N}(\alpha)=\frac{|H(\alpha)|^2}{(1/2\pi)\int\limits_{-\pi}^{\pi}
|H(\alpha)|^2 d\alpha}.
\end{equation}

\section{Classical Least-Square Algorithm}

The N-step least-squares phase-shifting algorithm formula was deduced in
1974 by Bruning et al. \cite{Bruning:74} following a synchronous detection 
technique for the phase-demodulation of temporal phase-shifted interferograms. 
Later in 1982 C. J. Morgan \cite{Morgan} demonstrated that this family of PSAs 
correspond to the principle of least-squares estimation under the presence of 
external pertur bations. Finally, J. E. Greivenkamp \cite{Greivenkamp} shown in 
1984 that this least-squares fit provides the better phase-estimation against 
non-uniform phase-steps.

In a PSI sequence each $(x,y)$ pixel is a 1D temporal discrete interferometric 
signal modeled as Eq. \eqref{eq:FringePatternTemporal}. Using the carrier 
$\c{C}(x,y,t)=\alpha k$, we have the following expression
\begin{eqnarray}
  I(x,y,k) &=& a(x,y) + b(x,y)\cos[\phi(x,y) +\alpha k] \nonumber \\
  &=& a(x,y) + c(x,y)\sin[\alpha k] - s(x,y)\cos[\alpha k],
  \label{eq:PSI_Sequence}
\end{eqnarray}
where $c(x,y)=b(x,y)\cos[\phi(x,y)]$ and
$s(x,y)=b(x,y)\sin[\phi(x,y)]$ are the quadrature components of the
1D temporal interferometric signal, $k$ is the discrete temporal
variable and $\alpha$ the phase step or temporal carrier; note that
all these variables are scalars. The independent temporal variable $k$
represents the $k$-frame of the PSI sequence. In this context, knowing
the phase step $\alpha$, the objective of the least-squares algorithm is to 
estimate the quadrature components $c(x,y)$ and $s(x,y)$ of the interferometric
signal at the $(x,y)$ pixel. Then, the phase at $(x,y)$ is obtained as 
mentioned above in Eq.\eqref{eq:PSIphase}.

Scanning all pixels in this way, we obtain the wrapped phase image of
the PSI sequence. One of the first approaches to demodulate a PSI
sequence was the least-squares model for PSI
\cite{Morgan,Greivenkamp,Okada,Kong}. The least-squares model (cost function) 
for PSI is the following,
\begin{align}\label{eq:PSIleast-squares}
  U[a,c,s]= \sum_{k=0}^{N-1}\left[a + c \sin(\alpha k)\right.
  -\left. s \cos(\alpha k)-I(k) \right]^2,
\end{align}
where $I(k)$ is the observed value of the $k$-frame at the
$(x,y)$ pixel modeled as in Eq. \eqref{eq:PSI_Sequence}. For clarity, 
the spatial dependence $(x,y)$ is omitted in parameters $a,c,s$ and $I$.
To have a well-posed mathematical model for Eq. \eqref{eq:PSIleast-squares},
it is necessary to have at least three interferograms in the PSI sequence;
that is, $N\geq 3$. The parameters $c(x,y)$ and $s(x,y)$ that minimize Eq. 
\eqref{eq:PSIleast-squares} are the quadrature components used in Eq. 
\eqref{eq:PSIphase} to obtain the phase. To minimize Eq. 
\eqref{eq:PSIleast-squares} we need to find
\begin{equation}
 \partial U/ \partial a(x,y)=0; \partial U/ \partial c(x,y)=0; 
\partial U/ \partial s(x,y)=0;
\end{equation}
that yields as solution
\begin{equation}
 X(x,y)=A^{-1}B(x,y)
\end{equation}
where $X(x,y)$ and $B(x,y)$ are a $3 \times 1$ vectors for every pixel given by
\begin{equation}\label{eq:LQX}
 X(x,y)=[a(x,y),c(x,y),s(x,y)]^T
\end{equation}
and
\begin{equation}
 B(x,y)=\bigg[\sum_k I(x,y,k),\sum_k I(x,y,k) \cos(\alpha k),
 \sum_k I(x,y,k) \sin(\alpha k) \bigg]^T
\end{equation}
and the matrix A does not depend on the position and is given by
\begin{equation}
A=\left[\begin{array}{ccc}
N & \sum_{k}cos(\alpha k) & \sum_{k}sin(\alpha k)\\
\sum_{k}cos(\alpha k) & \sum_{k}cos^{2}(\alpha k) & \sum_{k}cos(\alpha k)
\sum_{k}sin(\alpha k)\\
\sum_{k}sin(\alpha k) & \sum_{k}cos(\alpha k)\sum_{k}sin(\alpha k) & 
\sum_{k}sin^{2}(\alpha k)\end{array}\right].
\end{equation}
Matrix A needs at least three different phase shifts to be non-singular (and 
invertible) and make possible the computation of the solution $X(x,y)$ for every
pixel. From Eq. \eqref{eq:LQX} the wrapped phase for each pixel is calculated as
\begin{equation}
 \phi(x,y)=\arctan\left[\frac{-s(x,y)}{c(x.y)} \right].
\end{equation}
Matrix A must be calculated only once for frame, in consequece as the temporal 
interferometric signal of each pixel has the same model 
[see Eq. \eqref{eq:PSI_Sequence}], the solution of the linear equation system is 
always the same.

\section{Regularized Quadrature Filters (RQF)}

Regularization systems are very useful full-field systems
that can use all the information needed to obtain the data sought as
expected. In PSI, we can use these techniques to include the temporal
and spatial information to recover the modulating phase as a smooth
2D function, removing unwanted harmonics and noise. Actually,
regularization techniques have been used before in PSI for these
purposes, the first were Marroquin et al. \cite{RQF,AQF,AQF_mult} and more
recently others \cite{RPT, RQPT, Mariano, Mariano2, Vargas,Medina,Zeng}.









