\chapter{Fringe pattern demodulation}

\section{Introduction}

A fringe pattern is defined as a sinusoidal signal where a continuous map, 
analogous of the physical quantity being measured, is phase-modulated by an 
interferometer, Moire' system, etc. An ideal stationary fringe pattern is usually
modeled by
\begin{equation}
I(x,y)=a(x,y)+b(x,y)cos[\phi(x,y)],
\end{equation}\label{eq:FringePattern}
where $a(x,y)$ and $b(x,y)$ are the background and local contrast functions,
respectively; and $\phi(x,y)$ is the searched phase function.

Analyzing Eq. \ref{eq:FringePattern} one can see that the phase function
$\phi(x,y)$ cannot be directly estimated since it is screened by two other
functions, $a(x,y)$ and $b(x,y)$. Additionally, $\phi(x,y)$ can only be
determined modulo $2\pi$ because the sinusoidal fringe pattern $I(x,y)$ depends
periodically on the phase ($2\pi$ phase ambiguity); and its sign cannot be 
extracted from a single measurement without a priori knowledge (sign ambiguity)
due to even character of the cosine fuction [$cos(\phi)=cos(-\phi)$]. Finally, 
in all practical cases some noise $\eta(x,y)$ is introduced in an additive and/or
multiplicative way, and the fringe pattern may suffer from a number of distortions,
degrading its quality and further screening the phase information.




