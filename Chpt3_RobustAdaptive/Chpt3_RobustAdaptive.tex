\chapter{Robust adaptive phase-shifting demodulation for testing moving 
wavefronts}

\section{Abstract}
Optical interferometer setups are very sensitive when environment
perturbations affect its optical path. The wavefront under test is not
static at all. In this paper, it is proposed a novel and robust phase-shifting
demodulation method. This method locally estimates the interferogram’s
phase-shifting, reducing detuning errors due to environment perturbations
like vibrations and/or miscalibrations of the Phase-Shifting Interferometry
setup. As we know, phase-shifting demodulation methods assume that the
wavefront under test is static and there is a global phase-shifting for all
pixels. The phase-shifting demodulation method presented here is based on
local weighted least-squares, letting each pixel have its own phase-shifting.
This is a different and better approach, considering that all previous works
assume a global phase-shifting for all pixels of interferograms. Seeing this
method like a black box, it receives an interferogram sequence of at least 3
interferograms and returns the modulating phase or wavefront under test.
Here it is not necessary to know the phase shifts between the interferograms.
It does not assume a global phase-shifting for the interferograms, is robust
to the movements of the wavefront under test and tolerates
miscalibramiscalibrations of the optical setup with at least three
interferograms in the sequence.

\section{Introduction}
Phase-Shifting Interferometry (PSI) is a very known technique designed for
testing static wavefronts. In PSI, it is generated an interferogram sequence of
at least 3 interferograms with a phase shifting between them. To recover the
modulating phase the well known phase-shifting demodulation methods are used [1,
2, 3, 4, 5, 6]. When the wavefront under test remains static and the phase
shifts are introduced correctly, these phase-shifting methods recover the
modulating phase without error. However, in optical interferometer setups the
optical path is easily affected by environment perturbations. When environment
perturbations exist, the wavefront under test is moving and the phase-shifting
algorithms introduce an unavoidable detuning error [7, 8]. Nowadays, we can deal
with a moving wavefront by taking all the interferograms in the same instant of
time, for example, by using pixelated polarized cameras [10, 9]. But, today this
technology is expensive and is patent protected. To deal with the
miscalibrations of the optical set up, previous works propose self-tuning
phase-shifting demodulation methods [11, 12, 13]. However, all the previous
published works for PSI assume that the wavefront under test remains static and
there is a global phase-shifting for all pixels of the interferograms. This is
not true when environment perturbations affects the wavefront under test.
Suppose that you have a PSI optical setup, but, the wavefront under test is
perturbed by the environment in such a way that it is moving. The method
presented here is a robust adaptive phase-shifting demodulation method that let
us demodulate the interferogram sequence tolerating the movements of the
wavefront under test and miscalibrations from the PSI setup. This method allows
each pixel have its own phase-shifting, reducing considerably detuning errors
and improving the estimation of the modulating phase. To show the performance of
the Robust Adaptive Phase-Shifting (RAPS) method presented here, we will present
tests and results from simulated and experimentally obtained data.

\section{Method}
The interferometric phase-shifting signal for a single pixel has the classic
model found in all papers about phase-shifting. In that case, it is assumed that
all pixels has the same phaseshifting, therefore the interferometric signal for
any pixel is the following:
\begin{equation}
  I_k(x,y) = a(x,y)+b(x,y)cos[\theta_0(x,y) + \omega_0 k],
\end{equation}
where $I_k(x,y)$ is the $k$-interferogram of $M\times N$ pixels, $a(x,y)$ is its
background illumination, $b(x,y)$ its contrast or modulation term,
$\theta_0(x,y)$ is the wavefront under test and $\omega_0$ is the phaseshifting
introduced by the PSI system for all pixels. When the wavefront under test is
perturbed by environment, it is moving and the movements affects the phase shift
on each pixel of the interferograms in the following way:
\begin{equation}
  I_k(x,y) = a(x,y)+b(x,y)cos[\theta_0(x,y) + \eta_k(x,y) + \omega_0 k],
\end{equation}
where $\eta_k(x,y)$ is the environment perturbation. To process this
information, we are going to take $\eta_k(x,y)$ and $\omega_0$ as $\beta_k(x,y)
= \eta_k(x,y) + \omega_0 k$ in such a way that the interferogram sequence
can be rewritten as
\begin{equation}
  I_k(x,y) = a(x,y)+b(x,y)cos[\theta_0(x,y) + \beta_k(x,y)],
\end{equation}
where $\beta_k(x,y)$ represents the induced non-static phase-shifting variation
of the $k$-interferogram. Now, we are going to estimate the wavefront
$\omega_0(x,y)$ and the spatial $\beta_k(x,y)$ variations of each interferogram
of the sequence.

The Eq. (3.4) shows the least-squares cost function to recover the wavefront
under test $\omega_0(x,y)$; assuming that we know its spatial variations
$\beta_k(x,y)$ for each interferogram.
\begin{equation}
  E[a(x,y),f(x,y)]=\sum_{k=0}^{k-1}[a(x,y) + Re\{f(x,y) e^{i\beta_k (x,y)} \} -
I_k (x,y)]^2.
\end{equation}
In this equation, $i = \sqrt{-1}$ and $f(x,y)$ is a complex value for the
$(x,y)$ site. The operator $Re{\cdot}$ takes the real part of its argument,
that is, $Re{z} = \frac{1}{2} (z+z^*)$; being $z$ a complex value and $z^*$
its complex conjugate. $K$ is the number of interferograms. By minimizing (3.4)
with respect to $a(x,y)$ and $f(x,y)$, the wavefront under test (the modulating
phase) is recovered as
\begin{equation}
 \hat{\theta}(x,y)=angle[\hat{f}(x,y)],
\end{equation}
being $\hat{f}(x,y)$ the complex value that minimizes Eq. (3.4) at $(x,y)$ site.
To minimize Eq. (3.4) we solve the following linear system for each $(x,y)$
pixel:
\begin{equation}
\left(\begin{array}{ccc}
K & \sum c_{k}(x,y) & \sum s_{k}(x,y)\\
\sum c_{k}(x,y) & \sum c_{k}(x,y)^{2} & \sum c_{k}(x,y)s_{k}(x,y)\\
\sum s_{k}(x,y) & \sum c_{k}(x,y)s_{k}(x,y) & \sum s_{k}(x,y)^{2}
\end{array}\right)\left(\begin{array}{c}
\hat{a}(x,y)\\
\hat{\phi}(x,y)\\
\hat{\psi}(x,y)
\end{array}\right)=\left(\begin{array}{c}
\sum I_{k}(x,y)\\
\sum I_{k}(x,y)C_{k}(x,y)\\
\sum I_{k}(x,y)S_{k}(x,y)
\end{array}\right).
\end{equation}
The sums run from $k = 0$ to $K=-1$, being $K$ the number of interferograms.
$\hat{\phi}(x,y)$ and $\hat{\psi}(x,y)$ are the real and imaginary parts of 
$\hat{f}(x,y)$, respectively. $c_k(x,y)$ and $s_k(x,y)$ are the real and
imaginary parts of $e^{i\beta_k(x,y)}$, respectively.

Now, in the Eq. (3.7) we propose the local weighted least-squares cost function
for estimating the wavefront variations $\beta_k(x;y)$ for the $(x,y)$ site at
the $k$-interferogram; assuming that the wavefront under test $\theta_0(x,y)$ is
known.
\begin{equation}
E[a(x,y),g_{k}(x,y)]=\sum_{m=0}^{M-1}\sum_{n=0}^{N-1}\Bigg[\{a(m,n)+Re\{g_{k}(m,
n)e^ { i\theta_{0}(x,y)}\}-I_{k}(m,n)\}h(x-m,y-n)\Bigg]^{2}
\end{equation}


!!! CAMBIO DE PRUEBA !!!

