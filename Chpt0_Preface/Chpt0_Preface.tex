\chapter*{Preface}

The main objetive of this thesis is to present four new methods for optical 
interferometry images developed during the Ph.D. Three of these new methods were 
published in international journals; the other method listed in chapter 5, is in
the process of publishing.

In order to understand the new methods shown in this thesis we will present the
theorical principles of optical metrology as well as already published methods, 
same which would serve as basis for the new developed method presented here. 
The purpose to develop these methods was to solve common probles in interferometric
images. 

This thesis is organized as follows:

Chapter 1 introduces the fringe pattern demodulation in optical 
interferometry and reviews the theory behind the four new methods developed in 
this thesis.

Chapter 2 shows that the well known least-squares system for phase shifting 
interferometry (PSI) can be used as a full-field 2D linear system that uses the 
temporal and spatial information in conjunction in order to recover the modulating 
phase while removing noise, unwanted harmonics, and interpolating small empty
sections of the image space all in the same process with low computational time.

Chapter 3 develops a new regularization technique to demodulate a 
phase-shifting interferogram sequence with arbitrary inter-frame phase shifts. 
With this method, we can recover the modulating phase and inter-frame phase
shifts in the same process. As all phase-shifting algorithms, the assumption
is that the wavefront under testing does not change over time but, the phase-
shifting introduction can vary in a non constant way. A notable characteristic
of this demodulation method is that it not only can recover the modulating
phase, but also it is capable of filtering-out large quantities of corrupting
noise.

Chapter 4 proposes a novel robust phase-shifting demodulation method. 
This method locally estimates the interferogram's phase-shifting, reducing 
detuning errors due to environment perturbations like vibrations and/or 
mis-calibrations of the PSI setup. The phase-shifting demodulation method
presented here is based on local weighted least-squares, letting each pixel
have its own phase-shifting. This is a different and better approach, considering 
that all previous works assume a global phase-shifting for all pixels
of interferograms.

Chapter 5 offers an interesting method to remove the detuning distortions from 
the wrapped phase obtained by the uncalibrated phase interferometry demodulation 
methods. The method presented here takes the local frequencies as a priori 
knowledge from the wrapped phase, and uses an iterative approach to refine the 
wrapped phase.

Finally, I wish to acknowledge the financial support from the Centro de 
Investigaciones en \'Optica A. C. (CIO) and the Consejo Nacional de Ciencia y 
tecnolog\'ia (CONACYT). I also, want to thank my advisers Dr. Julio Estrada and
Dr. Manuel Servin by the guidence provided throughout my Ph.D.